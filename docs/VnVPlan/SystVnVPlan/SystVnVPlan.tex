\documentclass[12pt, titlepage]{article}

\usepackage{booktabs}
\usepackage{tabularx}
\usepackage{hyperref}
\hypersetup{
    colorlinks,
    citecolor=black,
    filecolor=black,
    linkcolor=red,
    urlcolor=blue
}
\usepackage[round]{natbib}

%% Comments

%\usepackage{color}
\usepackage[dvipsnames]{xcolor}

\newif\ifcomments\commentstrue

\ifcomments
\newcommand{\authornote}[3]{\textcolor{#1}{[#3 ---#2]}}
\newcommand{\todo}[1]{\textcolor{red}{[TODO: #1]}}
\else
\newcommand{\authornote}[3]{}
\newcommand{\todo}[1]{}
\fi

\newcommand{\wss}[1]{\authornote{blue}{SS}{#1}}
\newcommand{\an}[1]{\authornote{magenta}{Author}{#1}}
\newcommand{\meow}[1]{\authornote{Orchid}{JG}{#1}}


% jen things
\newcommand{\progname}{Kaplan} % PUT YOUR PROGRAM NAME HERE
\usepackage{amsmath} % n choose k becomes \binom{n}{k}
\usepackage{float} % for the [H] thing on tables and figures
\usepackage{textcomp} % degree symbol \textdegree
\usepackage{siunitx} % \si{\angstrom}

\begin{document}

\title{\progname{}: System Verification and Validation Plan} 
\author{Jen Garner}
\date{\today}
	
\maketitle

\pagenumbering{roman}

\section{Revision History}

\begin{tabularx}{\textwidth}{p{3cm}p{2cm}X}
\toprule {\bf Date} & {\bf Version} & {\bf Notes}\\
\midrule
Saturday 20 October, 2018 & 1.0 & Write initial draft (this time with a spell 
check for Texstudio...)\\
\bottomrule
\end{tabularx}

~\newpage

\section{Symbols, Abbreviations and Acronyms}

See the SRS document for \progname{} for other abbreviations and symbols, 
particularly see Section 2 (Reference Material).

\renewcommand{\arraystretch}{1.2}
\begin{tabular}{l l} 
  \toprule		
  \textbf{symbol} & \textbf{description}\\
  \midrule 
  T & Test\\
  FT & Functional Test \\
  NFT & Non-Functional Test \\
  QCM & Quantum Chemical Method \\
  BS & basis set \\
  SRS & Software Requirements Specification \\
  \bottomrule
\end{tabular}\\

\wss{symbols, abbreviations or acronyms -- you can simply reference the SRS
  tables, if appropriate}

\wss{You should comment out my original comments.  They are confusing now.}

\newpage

\tableofcontents

\listoftables

\listoffigures

\newpage

\pagenumbering{arabic}

The system verification and validation (systVnV) plan document is designed to 
outline the tests for \progname{} and show how the software will be satisfying 
its requirements. Verification means that the software was built correctly and 
validation means that the correct software was built (for the intended 
problem). The two important sections to this document include the planning 
section (\ref{section-plans}) and the testing section (\ref{section-tests}). In 
Section \ref{section-plans}, a brief overview as to how to verify the SRS, 
Design, and Implementation of \progname{} will be given. Furthermore, software 
validation techniques will be discussed. In Section \ref{section-tests}, 
specific test inputs will be provided, and the expected outputs for these tests 
will be described. The functional tests (FT) are designed to satisfy functional 
requirements (R). The non-functional tests (NFT) are designed to satisfy some 
of the non-functional requirements. All requirements are listed in the SRS 
documentation. Section \ref{section-tables} gives traceability for these 
requirements. A usability questionnaire is given in Appendix \ref{survey}.

\wss{provide an introductory blurb and roadmap of the
  Verification and Validation plan}

\section{General Information}

\subsection{Summary}

\progname{} is designed to search the potential energy space of molecules for 
possible conformational isomers. The inputs will be some geometry specification 
for the molecule(s) of interest, a quantum chemical method (QCM), and a basis 
set (BS). The output will be a geometry specification and a list of relevant 
energies. \wss{A geometry specification, or potentially a set of geometry specifications?}

\wss{Say what software 
is being tested.  Give its name and a brief overview of its general functions.}

\subsection{Objectives}

These tests should ensure that the input geometry can be converted to z-matrix 
format, such that the dihedral angles can be extracted for manipulation. The 
tests should also accurately procure a molecule's energy (which confirms that 
the input geometry converges with the given basis set and method). Once the 
conformational isomers have been found, the tests should certify that they are 
distinct from one another (and not merely a translation or rotation in space). 
The output geometries should be parsable by other quantum chemical software.

\progname{} intends to be simple in its use and upkeep. Any error messages 
should be verbose and steps to fix the issues will be given as suggestions. The 
maintainability is important, since the code must interact with many other 
libraries. If the code is not maintainable and the related libraries change, 
then it will be difficult and time consuming to perform updates. 

The tests should prove that \progname{} is robust with respect to a wide range 
of inputs, and (in passing) they should convince the user that their returned 
structures are chemically meaningful. The performance of \progname{} should be 
easily tracked and it should be evident to the user if their method and basis 
set are too computationally expensive for the input molecule.

\wss{State what is intended to be accomplished.  The objective will be around
  the qualities that are most important for your project.  You might have
  something like: ``build confidence in the software correctness,''
  ``demonstrate adequate usability.'' etc.  You won't list all of the qualities,
  just those that are most important.}

\subsection{Relevant Documentation}

The Software Requirements Specification (SRS) document included in the 
\progname{} repository should be referenced for any definitions. It may be 
necessary to reference the documentation for external software such as Psi4 
\cite{psi4}, Openbabel \cite{obabel}, \cite{obabel-web}, or Gaussian \cite{g16}.

\wss{Reference relevant documentation.  This will definitely include your SRS}

\wss{You should include the url for the other \progname{} documents.}

\section{Plan} \label{section-plans}
	
\subsection{Verification and Validation Team}

Mulder, c'est moi.

\wss{Probably just you.  :-)}

\subsection{SRS Verification Plan}

As stated in the SRS, two likely changes are present for \progname{}. First, 
LC1 states that the fitness function may not be a linear combination of the sum 
of energies and the sum of RMSD values. Once the code has been written and 
tested, then the fitness function can be modified to determine if a more 
efficient and/or accurate function exists. The modification will depend on the 
time available; the optimal fitness function is not expected to be found, but 
rather to be improved upon.

Second, LC2 highlights assumption A11, which states that the conformer space is 
independent of the ordering of atoms. If $n_a$ atoms are in molecule M, then 
$n_a-3$ dihedral angles are needed. There are a total of $n_a$ choose $n_a-3$ 
ways to select dihedral angles (simplified below in Equation 
\ref{choose-order}). The question to answer will be: does the set of chosen 
atoms change the conformational isomers that are found or the time needed to 
find them?

\begin{equation} \label{choose-order}
\binom{n_a}{n_a-3} = \frac{n_a!}{3!(n_a-3)!} = \frac{n_a(n_a-1)(n_a-2)}{6} 
\text{ ; } n_a > 3
\end{equation}

This question is relatively easy to test, as the input molecular geometry can 
be re-ordered such that different subsets of atoms are chosen to construct the 
z-matrix dihedral angles. It should be noted here that this test should be run 
for M with as little symmetry as possible ($C_1$ point group), to contract the 
search space and avoid comparing mirror images. The time to solution/energies 
(output) will be compared for M. If the results are not equal (when using the 
same random seed), then there is either a bug in the code, or the choice of 
dihedral angles matters. A different time to solution suggests that the energy 
calculations are somewhat dependent on the input configuration. A different 
energy implies that some dihedral angles are more important than others. The 
next question here would be - can a pattern be found to minimize the time to 
solution or find lower energy conformers? For example, does the set of dihedral 
angles, $D_i$, with heavier atoms converge more rapidly? Does $D_i$ outperform 
when atoms are chosen with more bonded atoms?

\wss{I like that you have thought about how your SRS might change as a result of
  your test cases.  Nice.}

As for the SRS document itself, the author will review it again once the rest 
of the program has been written to see if its requirements have been met. The 
author can ask her supervisor to review the paper as well as a few lab members. 
Brooks MacLachlan has already raised some github issues for the SRS document 
that the author will address by the end of this semester (Fall 2018). Professor 
Spencer Smith will also review the SRS and give feedback.

\wss{List any approaches you intend to use for SRS verification.  This may just
  be ad hoc feedback from reviewers, like your classmates, or you may have
  something more rigorous/systematic in mind..}

\subsection{Design Verification Plan}

The design will be reviewed during a meeting with the author's supervisor.
\wss{Your classmates and myself will also review  your design.}

\wss{Plans for design verification}

\subsection{Implementation Verification Plan}

This document will represent the system testing. Another document will be
written to outline the unit tests to be performed. \wss{You can actually point
  to the repo where this documentation will be located.} The code will be
written in Python, and nosetests \wss{Can you point to a url that defines
  nosetests?} will be used to run the tests. Various linters will be used, such
as pylint, pydocstyle, and pycodestyle, to ensure that standards are being
upheld for the code formatting and documentation (PEP8, numpy
docstrings). \wss{These ideas are good, but they aren't very specific.  I would
  rather you made your decisions here.  You can always go back and edit the
  document if you change your mind later.} The implementation will also be
tested by classmate, Malavika Srinivasan, and Professor Spencer Smith.

\wss{You should at least point to the tests listed in this document and the unit
  testing plan.}

\subsection{Software Validation Plan}

Journal articles that explore the conformer space of a given molecule will be 
used as a reference, for example see \cite{butane-conformers}. Using the same 
input molecule, determine whether 
\progname{} can find the same conformers.

\wss{Great!  Nice to see someone with a validation plan.}

\wss{If there is any external data that can be used for validation, you should
  point to it here.  If there are no plans for validation, you should state that
  here.}

\section{System Test Description} \label{section-tests}
	
\subsection{Tests for Functional Requirements}

\wss{Subsets of the tests may be in related, so this section is divided into
  different areas.  If there are no identifiable subsets for the tests, this
  level of document structure can be removed.}

\subsubsection{FT1 - Data Input and Output}

These tests will ensure that all data conversion and formatting is handled 
correctly by \progname{}. As per functional requirement R1, the user must 
provide 6 items to the program: $n_G$, $C_E$, $C_{RMSD}$, BS, QCM, and a 
molecular geometry. 

\begin{enumerate}

\item Non-Geometry Input Type Check

Description: test the inputs and check that they are of the correct type and 
have sensical bounds.

Control: Automatic

Initial State: empty, no molecule input.

Input:
\begin{table}[H]
	\begin{tabular}{p{3cm}p{3cm}p{2.5cm}p{4cm}}
		\toprule
		Input Variable & Special Cases & Usual Case & Error Cases \\
		\midrule
		$n_G$    & 0, 1        & 5*          & `a', ``bloop'', 1.0, 5.43, -5 \\
		$C_E$    & 0          & 0.5, 5        & `a', ``blip'', -10 \\
		$C_{RMSD}$ & 0        & 0.7, 2       & `a', ``banana'', -13 \\
		BS       & BS from file*  & ``cc-pVTZ'', ``STO-6G'' & 
		`o', ``not-a-basis-set'', unavailable basis (for atoms), unavailable 
		basis (for external program), 10, 0.3 \\
		QCM      & N/A           & ``HF'', ``CCSD'' & `o', ``not-a-method'', 
		unavailable method, 7, 
		0.25 
		\\
		\bottomrule
	\end{tabular}
\label{table-inputs}
\caption{Test cases for the input variables.}
\end{table}

* May need to issue a warning if the user provides a value for $n_G$ that is 
larger than 20 (or do stress testing to find the upper bound for
$n_G$). \wss{Rather than describe this as a note, it would be nice if you gave
  the full details of the stress testing you would run to determine $n_G$.  What
  would be the base input?  What would you vary?   How do you know when the
  value of $n_G$ is too big?}\\
** Depending on the external program used for energy calculations, a 
user-specified basis set may not be supported. Here is a 
\href{https://joaquinbarroso.com/2011/11/02/gen-keyword-gaussian/}{link} to a 
tutorial for how to make a basis set for Gaussian (using gen keyword) 
\cite{link-gen-basis}, \cite{g16}.

Output: Expect the program to raise an input error (type error) for the error 
cases. The coefficients ($C_E$ and $C_{RMSD}$) can be either a floating point 
or an integer value, but must not be negative. The special cases occur when 
there is a reduction of function calls (i.e. have $n_G$ = 0 means returning 0 
rather than calculating $S_E$ or $S_{RMSD}$). The BS and QCM chosen must be 
available in the external program (as used for energy calculations). Ensuring 
that the BS is applicable to the input molecule, and ensuring energetic 
convergence for the input molecule, is not covered by this test - see 
subsequent Test FT1-\ref{test-input-geom}.

\wss{I assume the information on invalid inputs is in your SRS.  You should
  reference your SRS here. Ideally,  you should point to the data  contraints
  table in that document.}

How test will be performed: Provide inputs from each category as per Table 
\ref{table-inputs} and ensure that they provide errors where expected and 
minimize computation by following protocol for special cases.

\item Test Input Geometry \label{test-input-geom}

Description: This test ensures that a molecule can be input in various ways 
(SMILES, xyz, z-matrix) and still produce the same geometry. Furthermore, it 
checks that the geometry is compatible with the chosen BS and QCM. If the 
geometry can converge during the optimization using that BS and QCM, then the 
inputs pass.

Control: Automatic

Initial State: N/A

Input: set of various test molecules (as SMILES, xyz, and z-matrix). Provide 
geometries that are purposefully bad (for example, a flat molecule that is 
supposed to be 3D). These geometries should not converge, no matter how the 
dihedral angles are manipulated. Provide a geometry for which the energy 
calculation will not converge, but can converge provided the dihedral angles 
are changed. Here, an example could be to superimpose two atoms that can 
otherwise be separated by free rotation. Note: depending on how robust the 
chosen external program is for energy calculations, this type of geometry may 
be hard to find. A last part of this test will be to input a geometry for which 
the chosen BS does not include one or more of the molecule's atoms. A 
good resource to check for this availability is the 
\href{https://bse.pnl.gov/bse/portal}{Basis Set Exchange}  
\cite{basis-set-exchange-ref1}, \cite{basis-set-exchange-ref2}, 
\cite{basis-set-exchange-web}.
 Some atoms, 
especially the heavier elements, do not have as many BS available as others 
(they simply have not been calculated yet).  \wss{You should be more specific
  about what the inputs will be.  You have to decide before you can run the
  tests, so why not decide now.  This comment about more specific inputs applies
elsewhere as well.  The description of an input might be complicated in your
case.  You can put the details in an Appendix, or just reference a file, maybe
in your repo, that has the geometry input information.}

Output: Errors are expected when all energies are calculated as zero (which 
will be the default value for non-converging calculations) - i.e. $S_E = 0$ for 
$n_G > 0$. Note: non-converging here does not mean that the external program 
fails (or runs out of computational time or resources, etc. - those are 
separate errors). Errors should also be raised when the BS is available, but 
there is an atom in the geometry for which the BS has not been parametrized. 
Given the same random seed and molecule, the results should be equal (within a 
small tolerance), regardless of whether the geometry is initialized as a SMILES 
string, set of xyz coordinates, or z-matrix. Note: this may be changed later if 
the SMILES strings give very different geometries to known experimental 
structures of molecules.

How test will be performed: This test will require running the program for the 
testing molecules, as described in the input section. It is only intended to 
satisfy the first part of R4, which requires that the energy calculations 
converge.

\item Test Output Generation

Description: as per R5, the program should generate a geometry from $D_i$ and 
the starting geometry.

Control: automatic

Initial State: an optimised input molecule versus an unoptimised input molecule.

Input: list of dihedral angles, $D_i$, for each conformer, alongside the 
starting geometry for that molecule.

Output: xyz coordinates for each conformer. 

How test will be performed: \progname{} will generate a new z-matrix file using 
the original geometry and the new dihedral angles. Then, it will convert this 
z-matrix to xyz coordinates using Openbabel and compare the outputs. For the 
optimised molecule, test that the geometry is different from the input. For the 
non-optimal molecule, test that the geometries are equal.
					
\end{enumerate}

\subsubsection{FT2 - Calculations}\label{syst-vnv-calcs}

\begin{enumerate}
	
	\item Calculate $Fit_G$
	
	Description: This test will be a simple check on the formula for $Fit_G$. 
	The $C_{RMSD}$ and $C_E$ coefficients will be tested such that their limits 
	will be sufficiently covered.
	
	Control: Automatic
	
	Initial State: $n_G$ = 0, 1, 2, 7, -10, 1.6
	
	Input: For each initial state, calculate $Fit_G$ for the parameters in 
	Table \ref{table-fitg}.

	\begin{table}[H]
		\begin{center}
		\begin{tabular}{lrrrr}
			\toprule
			Variable & \multicolumn{4}{c}{Values} \\
			\midrule
			$C_{RMSD}$ 	   & -1.4  & 0      & 1    & 0.5   \\
			$C_E$		   & -4    & 1      & 0    & 0.5   \\
			$S_E$ 		   & -120  & 500.24 & 23.4 & 400.3 \\
			$S_{RMSD}$ 	   & -1.5  & 4.3    & 2    & 8.3   \\
			\midrule
			Result $Fit_G$ & Error & 500.24 & 2    & 204.3 \\
			\bottomrule
		\end{tabular}
		\end{center}
	\caption{Example input and output of $Fit_G$ calculations for some values 
	of coefficients and summations.}
	\label{table-fitg}
	\end{table}
	
\wss{How do you know what the results should be in the last column?  Is this
  from another program?}
	Output: The value for $n_G$ (number of conformers being optimised) 
	starts at 0, representing the base case where no conformers are being 
	searched for. In this case, the result should be zero, and no functions 
	need to be called. This test ensures no divide-by-zero errors occur. The 
	second value ($n_G$ = 1) assumes one conformer. The expected behaviour is 
	calculate $Fit_G$ and assert that the value is equal to $E_i$ (the energy 
	of the conformer). Note: some errors may arise if the $S_{RMSD}$ value is 
	not correctly handled (i.e. index-out-of-range errors), since there is only 
	one molecule and the RMSD expects $n_G > 1$.
	
	The next two values for $n_G$ should be calculated as usual (one odd and on 
	even value). No errors are expected. The last two values are invalid inputs 
	for $n_G$ and should raise errors ($n_G$ cannot be a float and it cannot be 
	negative).
	
	For each value of $n_G$, 3 sets of coefficients will be tested. The chosen 
	values attempt to represent the limits of what should be calculated.
	
	How test will be performed: Run all input combinations with a test script 
	and pass to nosetests.
	
	\item Calculate $S_E$
	
	Description: Psi4 is the chosen quantum chemistry package that will be used 
	to run most energy calculations. More packages will be added later-on in 
	the project. In case updates to Psi4 change the output energies, these 
	tests will confirm that the same answer is being returned for the same 
	input calculation. The results can be compared to literature values (within 
	error tolerance).

	Control: Automatic

	Initial State: $n_G$ = 5

	Input: Set of input molecules (previously calculated using Gaussian), BS, 
	and QCM.

	Output: $S_E$ 

	How test will be performed: use Psi4 to calculate the energies of a set of 
	geometries. The geometries will have two types: (1) they are all 
	equivalent, and (2) they have been minimally randomized based on an initial 
	convergent geometry. Calculate the $S_E$. Ensure that the result is 
	positive and (if applicable) compare to previous test results. Compare the 
	results to the Gaussian output for the same BS and QCM and ensure that 
	value is within a tolerance.
	
	\item Calculate $S_{RMSD}$
	
	Description: The rmsd repository made by user charnley \\
	(https://github.com/charnley/rmsd)\\ will be used to calculate the RMSD 
	between two molecular geometries. This test should ensure that the 
	$S_{RMSD}$ value is calculated correctly.
	
	Control: automatic
	
	Initial state: $n_G$ = 0, 1, 2, 6
	
	Input: given molecule, $M$, rotate this molecule 90\textdegree, called 
	$M_r$. Translate M 2\si{\angstrom}, called $M_t$. 
	
	Output: $S_{RMSD}$
	
	How test will be performed: use rmsd repository to calculate the RMSD 
	between geometries. As with the energy calculations, there should be a test 
	done on equivalent geometries to ensure the RMSD is 0. The RMSD should also 
	be 0 for $M_r$ and $M_t$ when compared to $M$. The second set of tests 
	should be on geometries that have been changed by a small, random amount 
	(+/-0.2\si{\angstrom}). Since such a change should not impact the rotation 
	matrix, the RMSD can be calculated directly in a spreadsheet or other tool 
	for validation.

	
\end{enumerate}

\subsection{Tests for Nonfunctional Requirements}

\subsubsection{NFT1 - External Package Compliance Tests}

This section of testing ensures that \progname{} satisfies the non-functional 
requirement of working well with other quantum chemistry packages. These tests 
will show that the chosen quantum chemistry packages work as intended and 
within quantifiable limits.

\begin{enumerate}
	
\item Testing Formatting Conversion Error

Description: This test will determine the change in geometry that occurs when 
the molecule undergoes a formatting change. The conversion is performed using 
Openbabel software. The test will confirm that Openbabel is a suitable library 
to use for the program and quantify the rounding/floating point error 
associated with n conversions. This test is a form of 
stress testing. The author is open to suggestions as to how many conversions 
should be performed before considering the conversion to be stable. If this 
test passes, then the program will not need to keep re-reading the input 
geometry from a file nor maintain the original geometry as a variable in 
memory. The program will therefore be more robust and reliable upon quantifying 
the formatting change error.

Type: functional (automatic running, manual inspection)

Initial State: N/A

Input/Condition: \\
SMILES string for caffeine CN1C=NC2=C1C(=O)N(C(=O)N2C)C \\
Number of conversions $\rightarrow n = {1, 10, 20, 30..10000}$.

Output/Result: coordinates for caffeine after n conversions. Check $D_i$ each 
conversion and atom ordering. R5 should be satisfied here, in that the geometry 
can be regenerated from $D_i$.

How test will be performed: Starting with the SMILES string for caffeine, use 
Openbabel to convert this format to Cartesian coordinates. Then, convert the 
xyz coordinates to a z-matrix with Openbabel. One conversion counts as 
converting to and from a z-matrix. Complete n conversions and calculate the 
RMSD error between the initial and final xyz geometries. Plot the results 
with Matplotlib (Python library) for all n. Ensure that there is a tolerance 
for the number of conversions that \progname{} executes. Note: it may be 
necessary to test this tolerance for molecules of differing sizes. This test 
should also ensure that, after each conversion, the dihedral angles list $D_i$ 
and the atom ordering remain constant.

\item Testing Error between Quantum Chemical Packages

Description: In order to determine which packages will to interface with 
\progname{}, their energy calculation results will be compared for the same BS 
and QCM. The energies should be similar within a tolerance, and this tolerance 
should not impact the results of the conformer search.

Type: functional
					
Initial State: N/A
					
Input/Condition: set of molecules, BS, and QCM.
					
Output/Result: $E_i$ for each molecule.
					
How test will be performed: Gaussian and Psi4 are the first two programs that 
will interface with \progname{}. The difference in energy for the same 
calculation shall be documented. If there is a significant difference in 
results for any given BS or QCM, then these inputs may be removed as options 
from \progname{} (at least until the case can be determined).

\end{enumerate}

\subsubsection{NFT2 - Robustness Tests for $Fit_G$}

\begin{enumerate}
	\item Testing different $Fit_G$ functions.
	
	Description: Since the form of $Fit_G$ is likely to change (and will be 
	changed for research purposes), the program should be able to handle a 
	change in form for this function.
	
	Type: functional
	
	Initial state: N/A
	
	Input/condition: run all tests for \progname{} (that are dependent on the 
	calculation of $Fit_G$) such that the $Fit_G$ is evaluated as:
	
	\begin{enumerate}
		\item $\frac{C_E S_E}{C_{RMSD} S_{RMSD}}$
		\item $\frac{C_{RMSD} S_{RMSD}}{C_E S_E}$
		\item $C_E C_E^2 + C_{RMSD}\sqrt{S_{RMSD}}$
		\item $C_{RMSD} C_{RMSD}^2 + C_E\sqrt{S_E}$
		\item $C_E\ln{S_E} + C_{RMSD} e^{S_{RMSD}}$
		\item $C_{RMSD}\ln{S_{RMSD}} + C_E e^{S_E}$
	\end{enumerate}

	Output/result: returns the results of the tests involving
        $Fit_G$. \wss{what are the testes that involve $Fit_G$?.  Could  you
          reference the labels of these tests to make this unambiguous?} If any 
	tests failed, these tests should be investigated and fixed such that they 
	pass for the new function for $Fit_G$.
	
	How test will be performed: write the new $Fit_G$ functions in an external 
	file and use a test to call them. It might be easier to add the $Fit_G$ 
	function choice as an input parameter to the program (however this choice 
	may not be readily visible to the user).
	
\end{enumerate}

\subsubsection{NFT3 - Parallelizability and HPC Testing}

\begin{enumerate}
	\item Testing for n cores.
	
	Description: this test will quantify the performance increase that occurs 
	when the number of available computing cores is increased.
	
	Type: late-stage implementation (functional, manual observation)
	
	Initial state: no optimization has been performed on molecule, M.
	
	Input/condition: use n cores to optimize M, where n = 1, 2, 4, 8.
	
	Output/result: return the running time for each optimization. 
	
	How test will be performed: Plot the time to solution (y) with increasing n 
	(x) using Matplotlib. Inspect this plot to ensure that the slope of this 
	plot is not negative (i.e. increasing cores increases performance). This 
	will likely be part of a later phase of implementation and
        testing. \wss{I like that you are planning on plotting these results.}
	
	\item Testing on HPC cluster.
	
	Description: the program will be used primarily on a HPC cluster (for 
	example, Compute Canada's cluster, Graham). This test should require the 
	user to install the program on a HPC cluster and run the other tests.
	
	Type: manual installation, automatic running
	
	Input/condition: tests for \progname{}.
	
	Output/result: result of tests.
	
	How test will be performed: author will manually install the program on 
	Graham and run the tests. 
	
	
\end{enumerate}

\subsection{Traceability Between Test Cases and 
Requirements}\label{section-tables}

\wss{Provide a table that shows which test cases are supporting which
  requirements.}

\begin{table}[H]
	\begin{center}
	\begin{tabular}{|l|c|c|c|c|c|}
		\hline
		& R1 & R2 & R3 & R4 & R5 \\
		\hline
		FT1-1 & X  &    &    &    &    \\
		\hline
		FT1-2 & X  &    &    &    &    \\
		\hline
		FT1-3 &    &    &    &    & X  \\
		\hline
		FT2-1 &    & X  &    & X  &    \\
		\hline
		FT2-2 &    & X  & X  & X  &    \\
		\hline
		FT2-3 &    & X  & X  &    &   \\
		\hline
	\end{tabular}
\end{center}
\caption{traceability table for functional requirements and tests.}
\end{table}

\begin{table}[H]
	\begin{center}
	\begin{tabular}{|l|c|c|c|c|c|}
		\hline
		& NFR1 & NFR2 & NFR3 & NFR4 & NFR5 \\
		\hline
		NFT1-1 &      &      &      &      & X    \\
		\hline
		NFT1-2  &      &      &      &      & X    \\
		\hline
		NFT2-1 & X    &      &      &      &      \\
		\hline
		NFT3-1 &      &      & X    &      &      \\
		\hline
		NFT3-2 &      &      & X    &      &     \\
		\hline
	\end{tabular}
\end{center}
\caption{traceability table for non-functional requirements and tests.}
\end{table}

\section{Static Verification Techniques}

\wss{In this section give the details of any plans for static verification of
  the implementation.  Potential techniques include code walkthroughs, code
  inspection, static analyzers, etc.}

It is assumed that any external programs that are used by \progname{} have 
their own tests and that these tests will be run (and hopefully pass) on the 
applicable systems.

Planned external tests for these programs (note: cannot test Gaussian as it is 
not open-source):

\begin{itemize}
	\item Openbabel: https://openbabel.org/docs/dev/Contributing/Testing.html
	\item Psi4: http://www.psicode.org/psi4manual/1.2/testsuite.html
	\item rmsd: https://github.com/charnley/rmsd/blob/master/tests.py
\end{itemize}

The author will provide a code review for this project at a lab meeting. The 
other lab members will answer the usability survey questions as outlined in the 
appendix section \ref{survey}.
				
Travis will be used to verify the build when a commit is made to the 
\progname{} repository. Travis will attempt to install \progname{} as a Conda 
package inside a Conda environment. 

Here are some important style links:

\begin{itemize}
	
\item https://www.python.org/dev/peps/pep-0008/
\item https://pypi.org/project/pycodestyle/
\item 
https://sphinxcontrib-napoleon.readthedocs.io/en/latest/example\_numpy.html

\end{itemize}

\wss{You can remove this section and add the relevant material to Section 4.4.}

\bibliographystyle{plainnat}

%\bibliography{SRS}\\
\bibliography {../../../ReferenceMaterial/References}

\newpage

\section{Appendix}

This is where you can place additional information.

\subsection{Symbolic Parameters}

The definition of the test cases will call for SYMBOLIC\_CONSTANTS.
Their values are defined in this section for easy maintenance.

\subsection{Usability Survey Questions} \label{survey}

\wss{This is a section that would be appropriate for some projects.}

\begin{enumerate}
	\item Were you able to install the program using the instructions alone, or 
	did you require additional researching/help? If any steps were unclear, 
	please explain how they might be improved.
	\item Did the program accept your molecular input?
	\item Did the program return the same number of conformer geometries and 
	energies as specified by your choice of input for $n_G$?
	\item When you view these conformers (chimera, VMD, pymol, etc.), do their 
	geometries seem appreciably different from one another?
	\item Did your selection of BS or QCM  result in any convergence errors?
	\item What did you choose for your coefficients? Do you think that the 
	results can be improved with a different choice of $C_E$ and/or $C_{RMSD}$?
	\item Was the output delivered more quickly or more slowly than expected?
	
\end{enumerate}

\end{document}