\documentclass[12pt, titlepage]{article}

\usepackage{booktabs}
\usepackage{tabularx}
\usepackage{hyperref}
\hypersetup{
    colorlinks,
    citecolor=black,
    filecolor=black,
    linkcolor=red,
    urlcolor=blue
}
\usepackage[round]{natbib}

%% Comments

%\usepackage{color}
\usepackage[dvipsnames]{xcolor}

\newif\ifcomments\commentstrue

\ifcomments
\newcommand{\authornote}[3]{\textcolor{#1}{[#3 ---#2]}}
\newcommand{\todo}[1]{\textcolor{red}{[TODO: #1]}}
\else
\newcommand{\authornote}[3]{}
\newcommand{\todo}[1]{}
\fi

\newcommand{\wss}[1]{\authornote{blue}{SS}{#1}}
\newcommand{\an}[1]{\authornote{magenta}{Author}{#1}}
\newcommand{\meow}[1]{\authornote{Orchid}{JG}{#1}}


% jen things
%% Common Parts

\newcommand{\progname}{Kaplan}

\usepackage{listings} % for code lstlisting
\usepackage{amsmath, mathtools} % for using _\text in math
\usepackage{chngpage} % for stretching margins around large tables

\usepackage{xr} % for references to external documents
\externaldocument{../../SRS/SRS}
\externaldocument{../../VnVPlan/SystVnVPlan/SystVnVPlan}

\begin{document}

\title{Test Report: Kaplan} 
\author{Jen Garner}
\date{\today}
	
\maketitle

\pagenumbering{roman}

\section{Revision History}

\begin{tabularx}{\textwidth}{p{3cm}p{2cm}X}
\toprule {\bf Date} & {\bf Version} & {\bf Notes}\\
\midrule
December 10, 2018 (Monday) & 1.0 & write review \\
\bottomrule
\end{tabularx}

~\newpage

\section{Symbols, Abbreviations and Acronyms}

See the Reference Tables in the System Requirements Specification (SRS) 
(Section \ref{section-ref-tables}) documentation for the repository: 
\url{https://github.com/PeaWagon/Kaplan}.

\renewcommand{\arraystretch}{1.2}
\begin{tabular}{l l} 
  \toprule		
  \textbf{symbol} & \textbf{description}\\
  \midrule 
  T & Test\\
  NFR & Non-functional requirement \\
  R & Requirement \\
  HPC & high-performance computing \\
  FT & Functional test \\
  NFT & Non-functional test \\
  pmem & Population member \\
  BS & basis set \\
  QCM & quantum chemical method \\
  CI & continuous integration \\
  \bottomrule
\end{tabular}\\

\newpage

\tableofcontents

%\listoftables %if appropriate

%\listoffigures %if appropriate

\newpage

\pagenumbering{arabic}

This document is a review of the system tests that have been performed on 
\progname{}. This report is a partner to the Unit VnV Report that is located in 
the docs/VnVReport/UnitVnVReport directory.

\section{Review of Planning and Future Work}

This section will review the System VnV Plan (found in the SystVnVPlan Section 
\ref{section-plans}). Since not all testing has been done to validate and 
verify the code, this section serves the purpose of being a ``to-do" list for 
future work.

\subsection{SRS}

The plan is still to test various fitness functions to see if an optimal form 
exists (or at least to optimize the coefficients $C_E$ and $C_\text{RMSD}$) for 
a few given input molecules. The author's supervisor has provided feedback on 
the validity of the assumptions that were made in the SRS. This feedback can be 
found in the github issue tracker. From this feedback, dihedral angles are not 
fully representative of the molecule if taken from a zmatrix file. Therefore, 
the plan to test different orders of atomic inputs and search for conformers 
may be thrown out in favour of rewriting a module to use proper dihedral 
angles. Most of the SRS issues that were raised by Brooks MacLachlan were 
addressed and closed.

\subsection{Design}

Many changes were made to the module guide (MG) document, including closing 
github issues raised by classmates Malavika Srinivasan, Oluwaseun Owojaiye, and 
Karol Sekis. The original set of 11 modules was increased to 13 to avoid 
multiple secrets being contained by a single module - specifically, the pmem 
and geometry modules were added. The module instance specification (MIS) 
document must be changed such that the state variables reflect the definition 
of a state variable.

The design has yet to be reviewed by the author's supervisor; a review will be 
done after all github issues have been resolved from classmates and Dr. Spencer 
Smith. Still to implement in the design - use of fit\_form (ga\_input 
parameter) and prog (mol\_input parameter), since these are currently 
placeholder variables that do not do anything in the code.

The ring module will be updated (meaning its design will be updated in the MIS) 
to include extinction operators. These extinction operators are designed to 
return the ring to a more exploratory state once it becomes filled. The code 
has been written such that addition of these extinction operators will be 
relatively simple to implement. Some examples of extinction operators are:

\begin{itemize}
	\item \textbf{virus}: infect n parents and kill the segment +/- within 
	those parents (if parents have fitness lower than x).
	\item \textbf{plague}: kill lowest fraction of population (arranged by 
	fitness).
	\item \textbf{asteroid}: select random segment and delete all within that 
	segment.
	\item \textbf{deluge}: set a ``waterlevel" for the fitness; all below that 
	waterlevel are killed.
	\item \textbf{agathic}:	kill the oldest fraction of the population.
	
\end{itemize}

\subsection{Implementation}

Still to do for implementation verification is to fix all of the linting issues 
raised by the Travis build (see the UnitVnVReport for details). All tests for 
external libraries have been run, with the exception of rmsd and pubchempy. 
Other test results are discussed in Section \ref{section-auto-testing} of this 
document.

\section{Functional Requirements Evaluation}

R1 is the requirement that the program can read inputs:
\begin{itemize}
	\item $n_G$ is given as num\_geoms in the ga\_input module.
	\item $C_E$ is given as coef\_energy in the ga\_input module.
	\item $C_\text{RMSD}$ is given as coef\_rmsd in the ga\_input module.
	\item BS is given as basis in the mol\_input module.
	\item QCM is given as qcm in the mol\_input module.
	\item The molecular geometry is specified using struct\_input and 
	struct\_type in the mol\_input module. The charge and multiplicity of the 
	molecule are also given (as charge and multip) in the mol\_input module.
\end{itemize}

R2 is the requirement that the inputs can be used to solve for the fitness 
function $Fit_G$. The initial geometry should be used to generate potential 
solutions (in the form of sets of dihedral angles). The mol\_input and 
ga\_input modules both have verification functions to ensure that the inputs 
can be used to calculate the fitness function. Furthermore, the pmem module is 
a data structure that generates dihedral angles randomly (so that potential 
solutions to the conformer optimization are generated).

R3 requires the energy for each conformer and the rmsd for each conformer pair 
be calculated - an energy module and an rmsd module have been written for these 
purposes. The fitg module combines these values such that the fitness function 
can be calculated.

R4 requires that the energy calculations converge. Part of the verification of 
the mol\_input module is to ensure that the initial geometry can be run and 
that the given basis set and method are in the chosen program. $Fit_G$ is 
required to be positive, and this requirement is satisfied by making the 
coefficients positive and by forcing the user to use fit\_form = 0 (the same as 
the fitness instance model in the SRS).

R5 is that the output geometries are generated, which is satisfied by the use 
of the geometry module and the output module. External libraries that were used 
to help with this requirement include vetee, openbabel, and pubchempy. The ring 
data structure from the ring module has an attribute called zmatrix that keeps 
track of the original geometry specification, such that we can confirm the 
ordering has not changed.

R6 is that there are templates for the user to follow. They are provided in the 
test/testfiles directory and are called example\_mol\_input\_file.txt and 
example\_ga\_input\_file.txt.

R7 is that the conformers are optimized to maximize fitg. The tournament module 
selects a random number of solutions from the ring. Then it choses the two 
pmems from the tournament that have the highest fitness and uses those to 
generate new solutions for the ring. In this way, the value of fitg can become 
optimized, since only the best pmems are chosen for reproduction.

\section{Nonfunctional Requirements Evaluation}

NFR1 states that \progname{} should be robust with regards to changes made to 
the fitness function. The program has been written such that more fitness 
functions can be added via the fit\_form input variable in the ga\_input. The 
code maintainer would then have to add the fitness function to the fitg module 
and ensure that the inputs were constrained such that the fitness would be 
positive.

NFR2 states that the code should be maintainable. Having written the 
documentation in detail and provided a github repository should satisfy this 
requirement. The code has also been explained to two fellow group members in 
the author's lab, such that more people understand how the code should work.

NFR3 is that the program should be:
\begin{itemize}
	\item parallelisable (not yet satisfied under current conditions) - this 
	requirement could be satisfied by implementing python's multiprocessing 
	module
	\item capable of running on high performance computing servers (not yet 
	satisfied - in progress of being installed on Compute Canada's cluster, 
	graham)
	\item without a difficult install process - so far this program has been 
	successfully installed on 4 systems (Ubuntu on Windows, Ubuntu, Kbuntu) and 
	tested on Ubuntu.
\end{itemize}

Originally, the program was planned to be made into a conda package. This is 
still the plan, but for now it is installed using pip install (since the build 
is not yet passing).

NFR4 is that the program should be easy to use and quick to explain to 
chemists. A discussion of this requirement is covered by Section 
\ref{section-usability}.

NFR5 states that the program should work well with other quantum chemistry 
packages. Similar to the NFR1 regarding robustness with respect to fitness 
calculation, a placeholder input variable (called prog) in the mol\_input 
module will be used to specify the program the energy module should use to 
perform quantum chemical calculations. Right now, the only acceptable input is 
psi4, which was chosen to ensure that proprietary software was avoided (thus 
making the package easy to install for all users). In the future, Horton, 
Gaussian, and others may be added. These additions will change which functions 
are called from the energy module by the ring module (and more functions will 
therefore need be added to the energy module). Since the current state of the 
package can interact with openbabel, vetee, psi4, and pubchempy without issue, 
then it is considered to partially satisfy this non-functional requirement.

\subsection{Usability}\label{section-usability}

To make the program easy to use, the interaction with the source code has been 
removed. The user does not need to know any coding to run the program. Instead, 
the user must specify their inputs in two plain text files. The user can 
provide a variety of input types, which means the program can be used even if 
an exact geometry is not known (for example, a smiles string can be used to 
generate a geometry using openbabel).
		
\subsection{Performance}

The Travis Continuous Integration (CI) for \progname{} takes about 20 minutes 
to run (which includes installation and running the tests). The performance 
bottleneck is most likely the running of the psi4 program for the energy 
calculations. Performance is not yet a focus of the program.
	
\section{Comparison to Existing Implementation}	

There is no existing implementation of Kaplan, therefore this section is not 
relevant.

\section{Unit Testing}

The unit testing will be covered in more details in the UnitVnVReport. The 
actual implementation of the unit tests can be found in the kaplan/test/ 
directory on the github repo \url{https://github.com/PeaWagon/Kaplan}.

\section{Changes Due to Testing}

The actual System VnV Plan tests will be discussed henceforth. FT1-1 has been 
mostly satisfied by the unit tests that were written for the input modules. 
FT1-2 will be improved; currently the tests for checking basis set and method 
availability have been by a try accept loop in python where (if the external 
program throws an error with the inputs) \progname{} throws an error. Instead, 
a complete list of available basis sets and methods are being prepared such 
that the user of \progname{} can reference these documents. FT2-1 has not been 
completed, but will be added to the unit tests (so far only values between 0 
and 1 have been tested for the coefficients). FT2-2 and FT2-3 have been tested 
using jupyter notebooks (which in some cases have made it into the unit tests).

From the System VnV Plan, NFT1 can actually be disregarded. The original 
geometry is stored in the program to minimize the error when converting 
dihedral angles back into full geometry specifications. NFT2 has not yet been 
completed, but (as mentioned in the first paragraph of this section) the 
groundwork has been written into the code to enable such tests to be completed. 
NFT3 testing, as mentioned in the discussion of NFR3, for n cores can be 
completed when the code is parallelized. NFT3 testing on a HPC cluster can be 
completed once a successful install is performed on graham (still to do).

\section{Automated Testing}\label{section-auto-testing}

The testing was done using nosetests 
\url{https://nose.readthedocs.io/en/latest/}. A Travis CI build for the code 
was made by committing the repository containing a .travis.yml file. Each time 
a commit is made, the build is started and the author gets progress reports by 
email. Currently, the builds are ``failing", because none of the linters are 
passing. To see the builds, here is the link: 
\url{https://travis-ci.org/PeaWagon/Kaplan}.

The external testing for psi4 was run using nosetests. The test file was found 
in the psi4 directory for the kenv conda environment:

\begin{adjustwidth}{-0.5in}{-0.5in}
\begin{center}
\begin{lstlisting}
~/miniconda3/envs/kenv/lib/python3.6/site-packages/psi4/tests/test_psi4.py
\end{lstlisting}
\end{center}
\end{adjustwidth}

The results from running this test file have also been added to the \progname{} 
repository under kaplan/test/external-tests in a file called 
nose-psi4-test-result.txt. 2 out of 6 tests failed (errors for scf-property and 
dfmp2-1). Since the tests were run from a random file and not from the compiled 
version of the code, it is unknown as to their coverage and/or relevance.

For vetee, the tests were run using nosetests. The results from running these 
tests are in the kaplan/test/external-tests directory in a file called 
nose-vetee-test-result.txt. There were 5 failures over 43 tests total. Since 
this code is a work in progress by the author and another lab member, vetee is 
expected to improve its testing output.

For openbabel, the tests were run using make for openbabel version 2.4.1 that 
was compiled on the author's ubuntu machine. Of 160 tests, none failed. The 
results for the tests can be found under 
kaplan/test/external-tests/make-openbabel-test-result.txt.

It should be noted that psi4, openbabel, vetee, numpy, rmsd, and pubchempy are 
installed via a conda environment in order to run \progname{}. Therefore, these 
tests are not run using the same version that might be used for a separate 
installation. These results are meant to show that the external libraries in 
use have tests and that they can be run.

The tests for pubchempy can be found here: 
\url{https://github.com/mcs07/PubChemPy/blob/master/tests/test_requests.py}

The tests for rmsd can be found here:
\url{https://github.com/charnley/rmsd/blob/master/tests.py}
		
\section{Trace to Requirements}

This section will be covered in the Unit VnV Report.

\section{Trace to Modules}		

This section will be covered in the Unit VnV Report.

\section{Code Coverage Metrics}

Current code coverage metrics can be found in the Travis CI build. For example 
here: \url{https://travis-ci.org/PeaWagon/Kaplan/builds/465865809} we have a 
code coverage of:

\begin{lstlisting}[language=python, showstringspaces=false]

Name                   Stmts   Miss  Cover
------------------------------------------
kaplan/__init__.py        12      0   100%
kaplan/energy.py          33      1    97%
kaplan/fitg.py            30      1    97%
kaplan/ga_input.py        46      1    98%
kaplan/gac.py             25      5    80%
kaplan/geometry.py        56      2    96%
kaplan/mol_input.py       59      5    92%
kaplan/mutations.py       30      0   100%
kaplan/output.py          31      1    97%
kaplan/pmem.py            11      0   100%
kaplan/ring.py           109      7    94%
kaplan/rmsd.py            11      0   100%
kaplan/tournament.py      30      0   100%
------------------------------------------
TOTAL                    483     23    95%
------------------------------------------

\end{lstlisting}


\bibliographystyle{plainnat}

\bibliography{SRS}

\end{document}